\subsection{PickScore for W\"urstchen}
In their experiments, Pernias et al.~\cite{pernias2024wrstchen} are able to show
that their method yields a significant upgrade in computational costs and in
training time, which sits at a $8\times$ reduction compared to Stable Diffusion
2.1~\cite{rombach2023sd_2_1}. Another goal of the paper is to keep the level of
image complexity. In order to compare the level of complexity to other models,
Pernias et al. picked several evaluation methods. They used the metrics
Fr\'echet Inception Distance~\cite{heusel2018ganstrainedtimescaleupdate} (see
Section~\ref{sec:experiments:stabel_diffusion:selction_of_fid}), the Inception
Score~\cite{ding2021cogviewmasteringtexttoimagegeneration}, and the human preference imitating
PickScore. Furthermore, they conducted a study with human participants. All were
given a text-prompt and the images of W\"urstchen and another model to compare.
In experiments Pernias et al. found that W\"urstchen is capable to keep up or
even succeed over models of the same dimensions. While working with W\"urstchen
we found to W\"urstchen to perform worse than presented by the authors. Thus, we
conduct experiments of the PickScore~\cite{kirstain2023pickapic} to
prove the correctness of our own suspicion. In this section, we first describe
the workingw of PickScore, then we present the setup of our experiments and
after demonstrating our results we provide a discussion.

\subsubsection{Pick-a-Pic and PickScore}
In order to imitate a human user, in the last years language models were trained
to model the behavior of a user. For text-to-image models Kirstain et
al.~\cite{kirstain2023pickapic} saw a gap for datasets and models that
indicate human preference. Thus, Kirstain et al. introduce the dataset
Pick-a-Pic which consists of data point that each have a prompt two images, that
were generated using the prompt and the respective user preference. For
collecting the data the authors constructed the Pick-a-Pic Web App that enables
its users to enter text-prompts. From these prompts, different model such as
Stable Diffusion 2.1~\cite{rombach2023sd_2_1}, Dreamlike Photoreal 2.0 and
Stable Diffusion XL~\cite{podell2024sdxl}. Two of these images are then
presented, and the user decides which image fits better to the prompt. The
not-preferred image is then replaced by a new image and so on. Kirstain et al.
mention that their application was used by intrinsically-motivated users who
they argue to yield
This application
was used by user, that Kirstain et al. argue to be  who they
% Pick A Pic
% selections are made by intrinsically motivated users instead of paid annotators
% the authors of the paper suggest to eval models with Pick A Pic, as "which better represent what humans want to generate than mundane captions"

% Pick Score
% having established a dataset of images that are annotated by human preference -> model that predicts this
% based on CLIP architecture -> encodes image and text using transformer to a d-dimensional vector returning their inner product s(x, y)
% two images are compared by taking the scoring function of two images and their prompt another input is the preference distribution vector p
% the predicted preference hat p is calculated through a softmax-normalization of the scores of the two images
% the model is trained by minimizing the KL-divergence between the preferences.
% training objective is analogous to InstructGPT reward model objective
% pickscore is able to outperform other competitors and human experts

\subsubsection{Experiment description}
50 images from Pick a pic dataset validation
\subsubsection{Results}
\begin{table}[t]
    \caption{Examples of generated images, showing the ground truth, the HED image that was used as the input condition, PITI's result, ControlNet's result and lastly, ControlNet's result when using the prompt ``realistic, hd, photo''.}
    \label{tab:wuerstchen:results}
    \centering
    \begin{tabular}{cccccccc}
        \toprule
        \multicolumn{2}{c}{\textbf{MSCOCO Localized Narratives}} & \phantom{0} & \multicolumn{2}{c}{\textbf{PartiPrompt}} & \phantom{0} & \multicolumn{2}{c}{\textbf{Pick-A-Pic}}                                 \\
        \cmidrule{1-2}\cmidrule{4-5}\cmidrule{7-8}
        SD 1.4                                                   & SD 2.1      & \phantom{0}                              & SD 1.4      & SD 2.1                                  & \phantom{0} & SD 1.4 & SD 2.1 \\
        \midrule
                                                                 &             & \phantom{0}                              &             &                                         & \phantom{0} &        &        \\
        \bottomrule
    \end{tabular}
\end{table}