\subsection{PickScore for W\"urstchen}
In their experiments, Pernias et al.~\cite{pernias2024wrstchen} are able to show
that their method yields a significant upgrade in computational costs and in
training time, which sits at a $8\times$ reduction compared to Stable Diffusion
2.1~\cite{rombach2023sd_2_1}. Another goal of the paper is to keep the level of
image complexity. In order to compare the level of complexity to other models,
Pernias et al. picked several evaluation methods. They used the metrics
Fr\'echet Inception Distance~\cite{heusel2018ganstrainedtimescaleupdate} (see
Section~\ref{sec:experiments:stabel_diffusion:selction_of_fid}), the Inception
Score~\cite{NIPS2016_8a3363ab}, and the human preference imitating
PickScore. Furthermore, they conducted a study with human participants. All were
given a text-prompt and the images of W\"urstchen and another model to compare.
In experiments Pernias et al. found that W\"urstchen is capable to keep up or
even succeed over models of the same dimensions. While working with W\"urstchen
we found to W\"urstchen to perform worse than presented by the authors. Thus, we
conduct experiments of the PickScore~\cite{DBLP:journals/corr/abs-2305-01569} to
prove the correctness of our own suspicion. In this section, we first describe
the workingw of PickScore, then we present the setup of our experiments and
after demonstrating our results we provide a discussion.

\subsubsection{PickScore}
%
\subsubsection{Experiment description}
\subsubsection{Results}