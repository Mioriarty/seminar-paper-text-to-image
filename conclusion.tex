\section{Conclusion}
In this paper, we looked into three different works. First, we explored how
Stable Diffusion's performance boost, by leveraging the latent space rather than
working directly in pixel space, enables feasible image generation on a larger
scale. We re-conducted the FID score experiment which confirmed that the results
of Rombach et al.~\cite{rombach2022stablediffusion} are reproducible, indicating
a well-conducted experiment. Additionally, we examined the details and behavior
of the FID score, proposing an improved version, though further research is
needed to fully understand its behavior especially concerning generalization.\\

Secondly we investigated W\"urstchen~\cite{pernias2024wrstchen}, which promises
to improve training efficacy of latent diffusion models while keeping or
improving the image quality with latent diffusion models of the same dimensions.
Based on our own perception we stated the hypothesis that W\"urstchen performs
worse than the results of the PickScore calculated by Pernias et al. portrait.
Our results confirm our suspicion as the preference of W\"urstchen is lower
than demonstrated by Pernias et al. while still in range to align with their
conclusion to keep a level in image quality with other latent diffusion models
of the same size. To get more accurate reproducible results we recommend
multiple iterations of the experiments\\

Lastly\\

Moreover, we observed how text-to-image models are transforming a variety of industries, from game design to tourism. However, these models also present challenges for certain regions and demographics, raising important questions such as those surrounding copyright.